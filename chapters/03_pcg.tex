\chapter{PCG for (Vector-)OLE}
\label{chapter:PCGforVOLE}
In this chapter, we recall a PCG variant based on the $R^{c}$-LPN$_{\tau}$ assumption (cf. Figure \ref{fig:module-lpn-game}) first proposed by Boyle et al. in \cite{boyle2020efficient}. The PCG supports expansion to (Vector-)OLE correlations, giving it many applications in cryptographic constructions. We begin with a simplified PCG based on the $\texttt{Primal-LPN}$ assumption. We then show how this approach is adapted to achieve an efficient PCG based on $\texttt{Ring-LPN}$. Finally, we recall the concrete PCG construction based on $R^{c}$-LPN$_{\tau}$ for OLE \cite{boyle2020efficient} and show an adaption that realizes VOLE correlations.

\section{PCG based on Primal-LPN}
\label{sec:naivePCG}
We first introduce a naive approach under a simplified setting. This PCG is based on LPN over a Ring $\mathbb{Z}_p$, which is, in fact, equivalent to $\texttt{Primal-LPN}$ \cite{blum1993cryptographic}, but for simplicity we follow the $\texttt{Dual-LPN}$ assumption for this approach. We can do this, as $\texttt{Dual-LPN}$ itself is equivalent to $\texttt{Primal-LPN}$ \cite{couteau2021silver}. Recall that under $\texttt{Dual-LPN}$, we assume that two distributions
$$
\left\{ \mathbf{H}, \mathbf{H} \cdot \mathbf{e} \,\middle|\, \mathbf{H} \stackrel{\$}{\leftarrow} \mathbb{Z}_p^{m \times n}, \mathbf{e} \stackrel{\$}{\leftarrow} \left\{ \mathbf{v} \in \mathbb{Z}_p^n \,\middle|\, \|\mathbf{v}\|_0 = \tau \right\} \right\}
$$
are computationally indistinguishable if $\mathbf{e}$ is a $\tau$-sparse error vector. The intuition for realizing a PCG for an OLE correlation from this is as follows: both parties share the same $\texttt{Dual-LPN}$ instance, meaning the LPN matrix $\mathbf{H}$ is public. However, each party $\sigma \in\{0,1\}$ holds a unique, secret error vector $\mathbf{e}_\sigma$. This setup results in each party possessing a value $\mathbf{H}\cdot \mathbf{e}_\sigma$ that is inherently related to the other party's value through the shared matrix $\mathbf{H}$. We can leverage this relationship to securely ’combine’ the values in a way that resembles multiplication via tensor products. The parties can utilize a secret-sharing scheme for this, such that they get an additive share of the multiplication without revealing their error vectors, ultimately producing pseudorandom OLE correlations. In the following, this approach is presented more formally.

\subsubsection{Construction}
We start with the generation of the PCG seeds. A $\tau$-sparse error vector $\mathbf{e}_\sigma \in \mathbb{Z}_p^n$ is sampled for each party $\sigma\in\{0,1\}$. Then the tensor product $(\mathbf{e}_0 \otimes \mathbf{e}_1)$ is computed, which results in a ($n\times n$)-matrix consisting of $\tau^2$ non-zero elements. The matrix is transformed into vector notation $(e_0, ... e_{n^2-1})$, where each element represents a corresponding coordinate as depicted in Equation \ref{eq:pcgPrimalLpn}. By interpreting this vector as a multi-point function that maps the $\tau^2$ vector indices (of the non-zero elements) onto their respective values, Function Secret Sharing (FSS) (cf. Definition \ref{def:FSS}) can be used for implicitly secret-sharing the result of $(\mathbf{e}_0 \otimes \mathbf{e}_1)$ additively. The keys generated by \texttt{FSS.Gen} and the error vectors $\mathbf{e}_\sigma$ serve as the PCG seed.
\begin{equation}
\label{eq:pcgPrimalLpn}
\mathbf{e}_0 \otimes \mathbf{e}_1 = 
\begin{pmatrix}
e_{0} & \cdots & e_{n} \\
\vdots & \ddots & \vdots \\
e_{n^2-n} & \cdots & e_{n^2-1}
\end{pmatrix}
\overset{\texttt{FSS.Gen}}{\longrightarrow}
(k_0, k_1), \:\:\:
\begin{aligned}
& \mathbf{\tilde{z}}_0 = \texttt{FSS.FullEval}(0, k_0) = (e_{0}^0, \ldots, e_{n^2-1}^0) \\
& \mathbf{\tilde{z}}_1 = \texttt{FSS.FullEval}(1, k_1) = (e_{0}^1, \ldots, e_{n^2-1}^1)
\end{aligned}
\end{equation}

By evaluating their respective FSS key $k_\sigma$, each party obtains additive shares $e_x^\sigma$ of matrix/vector element $e_x$, such that $e_x = e_x^0 + e_x^1$ for $ x\in [n^{2}]$. Notice here, that $e_x$ itself is already multiplicative shared through $\mathbf{e}_\sigma$ as it stems from $\mathbf{e}_0 \otimes \mathbf{e}_1$ (Equation \ref{eq:pcgPrimalLPNFssSecretSharing}). For a full reconstruction of $\mathbf{e}_0 \otimes \mathbf{e}_1$, each party must query $\texttt{FSS.Eval}$ $n^2$ times, which we denote in simplified form by $\texttt{FSS.FullEval}$.

\begin{equation}
\begin{aligned}
\label{eq:pcgPrimalLPNFssSecretSharing}
\mathbf{e}_0 \otimes \mathbf{e}_1 &= \mathbf{\tilde{z}}_0 + \mathbf{\tilde{z}}_1 \\
&= \texttt{FSS.FullEval}(0, k_0) + \texttt{FSS.FullEval}(1, k_1) \\
&= (e_{0}^0, \ldots, e_{n^2-1}^0) + (e_{0}^1, \ldots, e_{n^2-1}^1) \\
&= (e_{0}^0+e_{0}^1, \ldots, e_{n^2-1}^0+e_{n^2-1}^1) \\
&= (e_0, ... e_{n^2-1})
\end{aligned}
\end{equation}
Each party now holds an multiplicative share of $(\mathbf{e}_0 \otimes \mathbf{e}_1)$ through $\mathbf{e}_\sigma$ and an additive share of the same operation through $\mathbf{\tilde{z}}_\sigma$. From this, the parties utilize the relation to the LPN matrix $\mathbf{H}$ to generate pseudorandom OLE correlations. By extending their values under LPN multiplicative share  $\mathbf{x}_\sigma$ and additive share $\mathbf{z}_\sigma$ (Equation \ref{eq:pcg_naive}) are obtained, such that  $\mathbf{x}_0 \otimes\mathbf{x}_1 = \mathbf{z}_0 + \mathbf{z}_1 $ (Equation \ref{eq:pcgNaiveProof}). 

\begin{equation}
\begin{aligned}
& \mathbf{x_0} = \mathbf{H}\cdot \mathbf{e}_0 \\
& \mathbf{x_1} = \mathbf{H}\cdot \mathbf{e}_1
\end{aligned}
\:\:\:\:\:\: 
\begin{aligned}
&\mathbf{z}_0 = (\mathbf{H}\otimes \mathbf{H}) \cdot \mathbf{\tilde{z}}_0 \\
& \mathbf{z}_1 = (\mathbf{H}\otimes \mathbf{H}) \cdot \mathbf{\tilde{z}}_1
\end{aligned}
\label{eq:pcg_naive}
\end{equation}

Notice that, at this point, the parties do not need to interact with each other. Additive share $\mathbf{z}_\sigma$ can be computed independently by using the previously computed additive shares of  $(\mathbf{e}_0 \otimes \mathbf{e}_1)$ and exploiting the distributive property of matrix multiplication over vector addition as shown in Equation \ref{eq:pcgNaiveProof}.

\begin{equation}    
\begin{aligned}
\mathbf{x_0} \otimes \mathbf{x_1} &= (\mathbf{H}\cdot \mathbf{e}_0)\otimes(\mathbf{H}\cdot \mathbf{e}_1)\\ 
&= (\mathbf{H}\otimes \mathbf{H})\cdot (\mathbf{e}_0 \otimes \mathbf{e}_1)\\
&=(\mathbf{H}\otimes \mathbf{H})\cdot (\mathbf{\tilde{z}}_0 + \mathbf{\tilde{z}}_1) \\
&=(\mathbf{H}\otimes \mathbf{H})\cdot \mathbf{\tilde{z}}_0 + (\mathbf{H}\otimes \mathbf{H})\cdot \mathbf{\tilde{z}}_1 \\
&= \mathbf{z}_0 + \mathbf{z}_1 
\label{eq:pcgNaiveProof}
\end{aligned}
\end{equation}


Note that both $\mathbf{x}_0$ and $\mathbf{x}_1$ are pseudorandom under the $\texttt{Dual-LPN}$ assumption. The construction therefore extends the $\tau$-sparse error vectors $\mathbf{e}_i$ to realize $n$ independent secret shared multiplications of pseudorandom elements as $\mathbf{x}_0\otimes\mathbf{x}_1=\mathbf{z}_0+\mathbf{z}_1$, thus realizing an OLE correlation such that each party $\sigma$ holds the pair $(\mathbf{x}_\sigma, \mathbf{z}_\sigma)$. Although we achieve a PCG for OLE with this construction, the approach remains inefficient: The computation of (1) the multiplication $H\cdot \mathbf{e}_i$, (2) the tensor $(\mathbf{e}_0 \otimes \mathbf{e}_1)$, and (3) $\texttt{FSS.FullEval}$ for evaluating each secret share is of quadratic complexity. Certain FSS schemes allow for more efficient full-domain evaluation (cf. Section \ref{subsec:treebasedDSPFsImpl}). However, the construction is still computationally expensive when $n$ is large, which is to be expected since the number of individual OLEs we extend must be as large as possible to compensate for the setup cost.

\section{PCG based on Ring-LPN}
\label{subsec:imrpoved_pcg}
Efficient polynomial algorithms can be employed to mitigate the computational limitations described above. Notice that any $n$ dimensional vector has a polynomial representation of degree \(n\), where each vector element serves as a coefficient for each monomial, with the element's index acting as the monomial's exponent. Applying this to the previous approach, we get the following: let the error vectors $(\mathbf{e}_0, \mathbf{e}_1)$ now be degree-$n$ $\tau$-sparse polynomials $(e_0(X), e_1(X))$. The multiplicative shares $(x_0, x_1)$ are derived independently from evaluating the respective error polynomial on position $x$. The additive shares $(z_0, z_1)$ are computed as a secret-shared evaluation of the polynomial product $e_0(X)\cdot e_1(X)$, which are then also evaluated at $x$.

\subsubsection{Construction}
Doing the above naively would break the security of the PCG since $(x_0, x_1)$ must remain indistinguishable from random. To achieve this, Boyle et al. \cite{boyle2020efficient} proposed an approach that exploits the $\texttt{Ring-LPN}$ assumption by defining $R_p:=\mathbb{Z}_{p}[X]/(F(X))$ to be a ring for a degree-$n$ polynomial $F(X)$. Thus, for a randomly sampled public polynomial $a \in R_p$ and sparse polynomials $e,f\in R_p$, we assume that the pair $(a, a\cdot e+f \texttt{ mod } F(X))$ is indistinguishable under $\texttt{Ring-LPN}$ (cf. Section \ref{prelim:ring_lpn}). From this, we construct the PCG: Given degree-$n$ $\tau$-sparse polynomials $(e_0, e_1)$ and $(f_0, f_1)$ each party $\sigma$ obtains $(e_\sigma, f_\sigma$) as their LPN secret. The parties additively secret share the multiplication of their secrets by utilizing FSS as depicted in Equation \ref{eq:ringLpnSecretMult}.  We stick to vector representation for simplicity (although dealing with polynomials). Notice that polynomial multiplication is equivalent to performing a tensor product of their polynomial vector representation. 
\begin{equation}
((e_0, f_0) \otimes (e_1, f_1)) = (e_0 \otimes e_1) + (e_0 \otimes f_1) + (f_0 \otimes e_1) + (f_0 \otimes f_1) \overset{\texttt{FSS.Gen}}{\longrightarrow} (k_0, k_1)
\label{eq:ringLpnSecretMult}
\end{equation}
Notably, \texttt{FSS.Gen} only needs to be called once by interpreting the addition of all tensored terms as the underlying multi-point function. The parties can then compute their share $\tilde{z}_\sigma =$ \texttt{FSS.FullEval}($\sigma, k_\sigma$). Combining these values additively recovers the product of their LPN secrets (Equation \ref{eq:ringLPNfulleval}).
\begin{equation}
\begin{aligned}
\tilde{z}_0 + \tilde{z}_1 = \texttt{FSS.FullEval}(0, k_0) + \texttt{FSS.FullEval}(0, k_1) = ((e_0, f_0) \otimes (e_1, f_1)) 
\label{eq:ringLPNfulleval}
\end{aligned}
\end{equation}
Each party $\sigma$ now holds a multiplicative share of $((e_0, f_0) \otimes (e_1, f_1))$ through their LPN secret $(e_\sigma, f_\sigma$), as well as an additive share of the same operation through  $\tilde{z}_\sigma$. From this, the parties utilize the public LPN parameter $a$ to generate their tuple $(x_\sigma, z_\sigma)$ as shown in Equation \ref{eq:imroved_pcg}.
\begin{equation}
\begin{aligned}
& x_0 = a\otimes e_0 + f_0 \\
& x_1 = a\otimes e_1 + f_1
\end{aligned}
\:\:\:\:\:\:
\begin{aligned}
& z_0 = ((1,a)\otimes(1,a)) \otimes \tilde{z}_0 \\
& z_1 = ((1,a)\otimes(1,a)) \otimes \tilde{z}_1 
\end{aligned}
\label{eq:imroved_pcg}
\end{equation}

These result in a single shared OLE correlation over $R_p$, as tuples $(x_\sigma, z_\sigma)_{\sigma\in\{0,1\}}$ are correlated, such that $x_0\otimes x_1 = z_0 + z_1$ as shown in Equation \ref{eq:RingLPNPCGProof}. Notice also that $x_0$ and $x_1$ are now pseudorandom under the $\texttt{Ring-LPN}$ assumption.
\begin{equation}
\begin{aligned}
x_0 \otimes x_1 &= (a \otimes e_0 + f_0) \otimes (a \otimes e_1 + f_1) \\
    &= a \otimes e_0  \otimes a \otimes e_1 + a \otimes e_0 \otimes f_1 + f_0 \otimes a \otimes e_1 + f_0 \otimes f_1 \\
    &= (a \otimes a) \otimes (e_0 \otimes e_1) + a \otimes (e_0 \otimes f_1)  + a \otimes (f_0 \otimes e_1)  + (f_0 \otimes f_1) \\
    &= (a \otimes a) \otimes (e_0 \otimes e_1) + (1 \otimes a) \otimes (e_0 \otimes f_1)  + (1 \otimes a) \otimes (f_0 \otimes e_1)  + (f_0 \otimes f_1) \\
    &= ((1,a) \otimes (1,a)) \otimes ((e_0, f_0) \otimes (e_1, f_1)) \\
    &= z_0 + z_1
\label{eq:RingLPNPCGProof}
\end{aligned}
\end{equation}

\subsection{Splitting the Ring through $F(X)$}
\label{subseq:realtiontofx}
To adapt this to a PCG for OLE over $\mathbb{Z}_p$, cipher-packing techniques from fully homomorphic encryption schemes can be used \cite{smart2014fully}. If the prime $p$ and the polynomial $F(X)$ are carefully chosen, it is possible to decompose $F(X)$ into $n$ linear factors modulo $p$ to get an isomorphism between $R_p$ and $n$ elements in $\mathbb{Z}_p$. This implies that any OLE over $R_p$ can be split up into $n$ individual OLEs over $\mathbb{Z}_p$. Specifically, Boyle et al. \cite{boyle2020efficient} analyze different choices for $F(X)$ and find that cyclotomic polynomials (cf. Section \ref{prelim:cyclotomicPolys}) are the most appropriate since they allow for fast multiplication in $R_p$ using Fast Fourier Transform (FFT). In order to compute the isomorphism, $n$ unique roots of unity ($\xi_0, ...,\xi_{n-1}$) are needed, such that for each root $\xi_l$ it holds that $F(\xi_l)=0$. Then, each party $\sigma$ can evaluate its pair of polynomials $(x_\sigma, z_\sigma)$ on $\xi_l$ to get its share of the $l$-th OLE correlation in $\mathbb{Z}_p$.

\subsubsection{Choosing a Cyclotomic Polynomial}
An interesting property of $N$-th cyclotomic polynomials (Equation \ref{eq:prelim_cyclotomic_poly}) is that when $N=\eta$ for $\eta$ being a power of two, they take on the following simplified form:

\begin{equation}
\label{eq:choosingfx}
\Phi_\eta(X)= X^\frac{\eta}{2} + 1
\end{equation}

This subset of cyclotomic polynomials is particularly well-suited for use in the \texttt{Ring-LPN} based PCG approach, as it splits completely into $\eta/2$ linear factors modulo $p$ and is easy to compute for any choice of $\eta$. For the construction described above, we require the ring $R_p$ to split into $n$ linear factors modulo $p$. Therefore, assuming $n$ is a power of 2, $F(X)$ is chosen as $\Phi_\eta(X) = \Phi_{2n}(X) = X^n + 1$.


\subsubsection{Computing the Roots of Unity}
What remains is to calculate the roots of unity for $\Phi_{2n}(X)$ so that for any $\xi_j \in (\xi_0, \ldots, \xi_{n-1})$ it holds that $\Phi_{2n}(\xi_j) = 0$. We achieve this by leveraging the concept of \textit{primitive roots of unity} in $\mathbb{Z}_p$. Recall that a primitive root of unity, $\omega$, in the finite field  $\mathbb{Z}_p$ (where $p$ is prime) satisfies Fermat's Little Theorem:

\begin{equation}
\label{eq:fermatLT}
\omega^{p-1} \equiv 1 \pmod{p}
\end{equation}

Further, let $s$ be the \textit{smooth order} (the product of distinct prime factors) of $p-1$.  We can then compute the multiplicative group generator, $g$, as:

\begin{equation}
g \equiv \omega^{\frac{p-1}{s}} \pmod{p}
\label{eq:multGroupGen}
\end{equation}

To calculate $n$ roots, $\xi_j \in \mathbb{Z}_p$ (for $j \in [N]$), we use the following equation, requiring $i$ to be odd with $i=2j+1$:

\begin{equation}
\label{eq:root_main_equasion}
\xi_j = \left(g^{\frac{s}{2n}}\right)^{i} = \left(g^{\frac{s}{2n}}\right)^{2j+1}
\end{equation}

\begin{proof}
We now prove that $\xi_j$ is indeed a root of unity for $\Phi_{2n}(X)$, meaning $\Phi_{2n}(\xi_j)=0$. First, we generalize Fermat's Little Theorem for multiples of $(p-1)$:

\begin{equation}
\label{eq:fermatGeneralized}
\omega^{n\cdot(p-1)} = \left(\omega^{(p-1)}\right)^n = \left(1\right)^n \equiv 1 \pmod{p}.
\end{equation}

The generalization stems from the fact that in a finite field $\mathbb{Z}_p$, every non-zero element has a unique multiplicative inverse, and the only self-inverse elements are $1$ and $-1$. Considering $\omega^{p-1}$ as the self-inverse element equivalent to $1$, it follows that $\omega^{\frac{p-1}{2}}$ must act as a multiplicative inverse to $-1$. This becomes evident when we consider $\omega^{p-1} = \omega^{\frac{p-1}{2}} \cdot \omega^{\frac{p-1}{2}} = 1$ from which we formulate Equation \ref{eq:rootMinusOne}:

\begin{equation}
\label{eq:rootMinusOne}
\omega^{(p-1)/2} \equiv -1 \pmod{p}.
\end{equation}

By integrating Equations \ref{eq:root_main_equasion}, \ref{eq:fermatGeneralized} and \ref{eq:rootMinusOne}, we demonstrate that $\Phi_{2n}(\xi_j) = (\xi_j)^n + 1 = 0$:

\begin{equation}
\begin{aligned}
\label{eq:rootProof}
(\xi_j)^n + 1 &= \left(\left(g^{\frac{s}{2n}}\right)^i\right)^n + 1 \\
   &=\omega^{\frac{p-1}{s} \cdot  \frac{s}{2n} \cdot i \cdot n} + 1 \\
   &=\omega^{\frac{p-1}{s} \cdot \frac{s}{2} \cdot i} + 1\\
   &=\omega^{\frac{p-1}{2} \cdot i} + 1\\
   &=\omega^{\frac{p-1}{2} \cdot (2j + 1)} + 1\\
   &=\omega^{j \cdot (p-1)} \cdot \omega^{(p-1)/2} + 1\\
   &=1 \cdot -1 + 1 = 0
\end{aligned}
\end{equation}

This concludes our proof, showing that Equation \ref{eq:root_main_equasion} generates roots of unity for $\Phi_{2n}(X)$ under the finite field $\mathbb{Z}_p$. 
\end{proof}

\subsection{Element of Unity in $a$}
Note that Equation \ref{eq:imroved_pcg} includes '$1$' in the tensor product of the public LPN polynomial $a$. The reason for this is that in this notion, $e_0$ and $e_1$ technically serve as the LPN secret with $f_0$ and $f_1$ being the noise. In practice, the LPN secret and the noise can be combined into a single polynomial by fixing a position (or coefficient in the context of \texttt{Ring-LPN}) of $a$ to $1$. This approach becomes more clear when looking at the \texttt{Module-LPN} assumption and the $R^{c}$-LPN$_{\tau}$ security game (Figure \ref{fig:module-lpn-game}) which represents a generalization of \texttt{Ring-LPN}. Given $c$ randomly sampled ring elements $\tilde{e}_0, \tilde{e}_1, ... \tilde{e}_{c-1}$ and $\tilde{a}_0, \tilde{a}_1, ...\tilde{a}_{c-1}$ for $\tilde{a}_{c-1}=1$  the challenger computes the inner product $\tilde{u}_1$ as

\begin{equation}
\label{eq:inner_product_with_a_1}
\tilde{u}_{1} \leftarrow \sum_{i=0}^{c-1} \tilde{a}_{i} \cdot \tilde{e}_{i} = \sum_{i=0}^{c-2} \tilde{a}_{i} \cdot \tilde{e}_{i} + \tilde{e}_{c-1}
\end{equation}

while $\tilde{u}_0$ is sampled uniformly at random. Any distinguisher, given access to $\tilde{a}_0, \tilde{a}_1, ..., \tilde{a}_{c-1}$, is considered to win if it can distinguish between $\tilde{u}_0$ and $\tilde{u}_1$ with a non-negligible probability. The challenger wins, if the challenger is not able to do so. Through setting $\tilde{a}_{c-1}$ to $1$ we implicitly utilize $\tilde{e}_{c-1}$ as the LPN noise.


\subsection{Complexity Improvements}
\label{subsec:VOLEConstruction}
The \texttt{Ring-LPN} based PCG improves upon the quadratic complexity of the naive \texttt{Primal-LPN} based approach presented Section \ref{sec:naivePCG}. Specifically, the computation of $\texttt{FSS.FullEval}$ is now in $O(n)$ since the domain size is limited to $[0, ..., 2n-1]$. Additionally, polynomial multiplication, which is equivalent to computing the tensor product of two vectors, allows for the utilization of fast multiplication algorithms such as FFT. Hence, when FFT is used, the quadratic cost of a tensor product is reduced to a polynomial multiplication of superlinear complexity in $O(n \log n)$.

\section{Construction}
\label{sec:construction}
In the following, we present a concrete PCG construction by Boyle et al. \cite{boyle2020efficient} that builds on the notion of the \texttt{Ring-LPN} based PCG discussed above. For this construction, the \texttt{Ring-LPN} assumption is generalized to \texttt{Module-LPN}, which we denote with $R^{c}$-LPN$_{\tau}$ as formalized in Figure \ref{fig:module-lpn-game}. 
\\\\
The main building block of this construction is a Function Secret Sharing (FSS) scheme (cf. Definition \ref{def:FSS}). The FSS primitive must be able to realize secret sharing of the multiplication of degree-$N$ $\tau$-sparse polynomials. The result of such a multiplication is always a polynomial of degree $< 2N$ consisting of $\tau^2$ non-zero coefficients. In vector notation, this results in a $2N$-vector with $\tau^2$ non-zero elements. This vector can be interpreted as a multi-point function such that the positions of the non-zero elements in the vector are mapped to their respective value. Therefore, the function has a domain $[2N]$ mapping onto $\tau^2$ non-zero elements. As it's FSS scheme, this construction chooses distributed sum of point functions (DSPFs) for secret sharing the described multi-point functions. Under the derived parameters the DSPF is denoted as $\texttt{DSPF}^{\tau^2}_{2N}$. A formal definition for DSPFs is provided in Definition \ref{def:dspf}. The concrete PCG construction by Boyle et al. \cite{boyle2020efficient} (although adapted to our notation) is presented in the following:

\begin{specialconstruction}{PCG for Oblivious Linear Evaluation (OLE)}
\label{construction:PCGforOLEImpl}
\vspace{1em} 

Let $\lambda$ be the security parameter, $(c,\tau)$ the parameters of the $R^c$-LPN$_\tau$ assumption, and $p$ the modulus. We denote $N$ as the domain of the PCG. Let $R_p:=\mathbb{Z}_{p}[X]/(F(X))$ be a ring for a degree-$N$ polynomial $F(X) \in \mathbb{Z}_{p}[X]$ and ($\texttt{DSPF}^{\tau^2}_{2N}$.Gen, $\texttt{DSPF}^{\tau^2}_{2N}$.Eval) be a FSS scheme for multi-point functions that map a domain $[2N]$ onto $\tau^2$ non-zero elements. Further, let $\boldsymbol{a} = (1, a_2, ..., a_c)$ for $a_2, ...,a_c \in R_p$ be a public input.

\vspace{1em} 

PCG.Gen$_{\texttt{OLE}}(1^\lambda)$:
\begin{algorithmic}[1]
\State For every $r \in [c]$ sample $\boldsymbol{\alpha}_{0}^{r}, \boldsymbol{\alpha}_{1}^{r} \stackrel{\$}{\leftarrow}[N]^{\tau}$ and $\boldsymbol{\beta}_{0}^{r}, \boldsymbol{\beta}_{1}^{r} \stackrel{\$}{\leftarrow} [\mathbb{Z}_{p}]^{\tau}$ uniformly at random.
\State For $i, j \in [c]$, sample FSS keys:
\begin{align*}
& \left(K_{0}^{(i, j)}, K_{1}^{(i, j)}\right) \stackrel{\$}{\leftarrow} \texttt{DSPF}^{\tau^2}_{2N}\texttt{.Gen}\left(\mathds{1}^{\lambda}, \boldsymbol{\alpha}_{0}^{i} \boxplus \boldsymbol{\alpha}_{1}^{j}, \boldsymbol{\beta}_{0}^{i} \otimes \boldsymbol{\beta}_{1}^{j}\right).
\end{align*}
\State For $\sigma \in \{0,1\}$ set $k_\sigma = (\{\boldsymbol{\alpha}_\sigma^i, \boldsymbol{\beta}_\sigma^i\}_{i\in[c]}, \{K_\sigma^{i,j}\}_{i,j\in[c]})$.
\State Output $(k_0, k_1)$.
\end{algorithmic}

\vspace{1em} % Space before the next part

PCG.Expand$_{\texttt{OLE}}(\sigma, k_\sigma)$:
\begin{algorithmic}[1]
\State Parse $k_\sigma $ as $(\{\boldsymbol{\alpha}_\sigma^i, \boldsymbol{\beta}_\sigma^i\}_{i\in[c]}, \{K_\sigma^{i,j}\}_{i,j\in[c]})$.
\State For $i \in [c]$, define the degree $< N$, $\tau$-sparse polynomial:
\begin{align*}
e^i_\sigma(X) = \sum_{k\in [t]} \boldsymbol{\beta}_\sigma^i[k] \cdot X^{\boldsymbol{\alpha}_\sigma^i[k]}
\end{align*}
and compose all $e^i_\sigma$ to a single length-$c$ vector $\boldsymbol{e}_\sigma$.
\State For $i, j \in [c]$ compute vector $\boldsymbol{u}_\sigma$ with each element being interpreted as a degree $< 2N$ polynomial:
\begin{align*}
& \boldsymbol{u}_\sigma[i+c(j-1)] \leftarrow \texttt{DSPF}^{\tau^2}_{2N}\texttt{.FullEval}(\sigma, K_\sigma^{i,j}) \texttt{ mod } F(X).
\end{align*}
\State Compute polynomials $x_\sigma = \langle\boldsymbol{a},\boldsymbol{e}_\sigma\rangle$ \texttt{mod} $F(X)$ and $z_\sigma = \langle\boldsymbol{a}\otimes\boldsymbol{a},\boldsymbol{u}_\sigma\rangle$ \texttt{mod} $F(X)$.
\State Output $(x_\sigma, z_\sigma)$.
\label{c:OLEBoyle}
\end{algorithmic}
\end{specialconstruction}

\vspace{1em}

\textbf{Seed Generation.} For each participant $\sigma \in \{0,1\}$ the construction samples $c$ $\tau$-sparse vectors $\boldsymbol{\alpha}_\sigma$ as exponents and $c$ $\tau$-sparse vectors $\boldsymbol{\beta}_\sigma$ as coefficients representing the participant's respective LPN error vectors. Through sampling the coefficients and exponents separately the polynomial multiplication can be directly secret shared as a point function such that the outer sum of the exponents $\boldsymbol{\alpha}_{0}^{i} \boxplus \boldsymbol{\alpha}_{1}^{j}$ represent the \textit{special positions} and the tensor product of the coefficients $\boldsymbol{\beta}_{0}^{i} \otimes \boldsymbol{\beta}_{1}^{j}$ represent the respective \textit{non-zero elements}. Ultimately PCG.Gen$_{\texttt{OLE}}$, outputs seeds including error polynomials $(\{\boldsymbol{\alpha}_\sigma^i, \boldsymbol{\beta}_\sigma^i)\}_{i\in[c]}$ and the secret sharing of their tensors in form of FSS keys $\{K_\sigma^{i,j}\}_{i,j\in[c]}$  for each party. 
\\\\
\textbf{Seed Expansion.}
For expanding the PCG seeds in PCG.Expand$_{\texttt{OLE}}$ every party inputs its PCG seed and parses the components. The exponent and coefficient vectors are recombined to polynomial representation to compose then a length-$c$ vector $\boldsymbol{e}_\sigma$ of those. Further, each party computes its shares of the tensor products by fully evaluating their part of the DSPF. Recall that the party does not learn anything about the other party's error vectors (and, therefore, the shared multiplication) through this evaluation. The result of each full-domain evaluation is a $<2N$ degree polynomial, which is stored as an element of the $c^2$-length vector $\boldsymbol{u}_\sigma$. Analogue to Equation \ref{eq:imroved_pcg}, each party computes their polynomial pair $(x_\sigma , y_\sigma)$ by computing the inner products $x_\sigma = \langle\boldsymbol{a},\boldsymbol{e}_\sigma\rangle$ and $z_\sigma = \langle\boldsymbol{a}\otimes\boldsymbol{a},\boldsymbol{u}_\sigma\rangle$ which results in a single OLE over $R_p$. As described in Section \ref{subsec:imrpoved_pcg} and shown in Equation \ref{eq:inner_product_with_a_1}, the error vectors implicitly include the LPN secret at the first position which is preserved during the calculation of the inner product by fixing an element of the public input $\boldsymbol{a}$ to 1. 
\\\\
In terms of security, the following theorem is presented, with a proof available in \cite{boyle2020efficient}.
\begin{theorem}
Suppose the $R^c$-LPN$_\tau$ assumption holds and \texttt{DSPF} is a secure FSS scheme as formalized in Definition \ref{def:FSS}. Then, Construction 1 is a secure PCG for OLE in $R_p$ to Definition \ref{def:PCGprelim}.
\label{theorem:olepcg}
\end{theorem}

\subsubsection{Splitting the Ring}

Holding the polynomial pairs $(x_\sigma , y_\sigma)$ the parties can then exploit the isomorphism between $R_p$ and $\mathbb{Z}_{p}$ to generate $N$ individual OLE correlations in $\mathbb{Z}_{p}$ as described in Section \ref{subseq:realtiontofx}. The parties evaluating their pair on the same roots of unity of $F(X)$, such that for any root of unity $\xi_i \in \{\xi_0, ...,\xi_{N-1}\}$

\begin{equation}
x_0(\xi_i) \cdot x_1(\xi_i) = z_0(\xi_i)+ z_1(\xi_i)
\end{equation}
is the $i$-th OLE correlation in $\mathbb{Z}_{p}$.

\subsubsection{Computational Complexity}
The computational complexity of this construction depends on the LPN security parameters $(c, t)$ and the chosen domain $N$. In practice, the LPN parameters are rather small compared to $N$. For example, Boyle et al. \cite{boyle2020efficient} report $(c,\tau) \in \{(4,16),(8,5),(2,76)\}$ to achieve 128-bit security under the $R^c$-LPN$_\tau$ assumption. $N$ is expected to be much larger, since it is desirable to get as many OLE correlations from a PCG instantiation as possible. Regarding a specific DSPF construction we choose the tree-based DSPF construction of \cite{boyle2016function} for which \texttt{DSPF$^{\tau}_{N}$.Gen} is in $O(t)$ and \texttt{DSPF$^{\tau}_{N}$.FullEval} is in $O(N)$. Multiplying degree-$N$ polynomials via FFT is in $O(N\log N)$.
\\\\
The computational complexity of PCG.Gen$_{\texttt{OLE}}$ is in $O(c^2\tau^2)$ as the costliest operation is the generation of the DSPF seeds. For the outer sum of $\boldsymbol{\alpha}$, we require $\tau^2$ additions, and for the tensor product of $\boldsymbol{\beta}$, we require $\tau^2$ multiplications. \texttt{DSPF$^{\tau^2}_{2N}$.Gen} is in $O(\tau^2)$. As this is computed $c^2$ times the overall complexity comes to $O(c^2\tau^2)$. Although we have quadratic complexity here, this is not a bottleneck in practice as $(c,\tau)$ are small.
\\\\
The computational complexity of PCG.Eval$_{\texttt{OLE}}$ is in $O(N\log N \cdot c^2)$ for which the computation of $z_\sigma$ represents the main computational burden. Mainly, the use of FFT allows the complexity to stay superlinear for $N$. Using naive multiplication here would result in quadratic complexity for $N$, making the scheme undesirable for expanding large amounts of OLE correlations. The complexity can be derived from

\begin{itemize}
\item step 3, which is in $O(Nc^2)$, as we call \texttt{DSPF$^{\tau^2}_{2N}$.FullEval} $c^2$ times.
\item step 4, which is in $O(N\log N \cdot c^2)$, as we compute inner products for
\begin{itemize}
\item $x_{\sigma}$ which requires $c-1$ multiplications of degree-$N$ polynomials.
\item $z_{\sigma}$ which requires $c^2-1$ multiplications of dense degree-$N$ polynomials via FFT.
\end{itemize}
\end{itemize}

Notice, that using FFT to compute $x_{\sigma}$ is not beneficial for performance. The reason is that $\boldsymbol{e}$ contains $\tau$-sparse polynomials. Therefore, naive multiplication with degree-$N$ polynomials in $\boldsymbol{a}$ requires only $\tau\cdot N$ operations, which is in $O(N)$ instead of $O(N\log N)$ when using FFT for this case. We analyse this effect more thoroughly in Section \todo X.

\subsection{Adaption to VOLE}
We present a PCG construction for Vector-OLE (VOLE) correlations in the following. This construction is derived from slightly adapting the PCG for OLE (Construction 1) described above. Recall that VOLE is an special case of OLE for $N$-dimensional vectors ($\mathbf{x}_0$, $\mathbf{z}_0$, $\mathbf{z}_1$) and constant $v$ such that $\mathbf{x}_0\cdot v = \mathbf{z}_0 + \mathbf{z}_1$ holds $i \in [n]$ OLE tuples $\{(\mathbf{x}_0[i], \mathbf{z}_0[i]),(v, \mathbf{z}_1[i])\}$. 
\\\\
The intuition for this PCG is as follows: where we had to secret share the multiplication of two polynomials before, we now need to secret share the multiplication of a polynomial by a constant. As this does not change the degree of the polynomial, the result of the operation under a degree-$N$ $\tau$-sparse polynomial can be represented by a multi-point function of domain $N$ mapping onto $\tau$ non-zero elements. Therefore, we can realize a secret sharing using $\texttt{DSPF}^{\tau}_{N}$. In the following, we formalize the adapted construction.

\begin{specialconstruction}{PCG for Vector Oblivious Linear Evaluation (VOLE)}
\label{construction:PCGforVOLEImpl}

\vspace{1em}

Let $\lambda$ be the security parameter, $(c,\tau)$ the parameters of the $R^c$-LPN$_\tau$ assumption, and $p$ the modulus. We denote $N$ as the domain of the PCG. Let $R_p:=\mathbb{Z}_{p}[X]/(F(X))$ be a ring for a degree-$N$ polynomial $F(X) \in \mathbb{Z}_{p}[X]$ and ($\texttt{DSPF}^{\tau}_{N}$.Gen, $\texttt{DSPF}^{\tau}_{N}$.Eval) be a FSS scheme for multi-point functions that map a domain $[N]$ onto $\tau$ non-zero elements. Further, let $\boldsymbol{a} = (1, a_2, ..., a_c)$ for $a_2, ...,a_c \in R_p$ be a public input.

\vspace{1em}

PCG.Gen$_{\texttt{VOLE}}(1^\lambda, v)$:

\begin{algorithmic}[1]
\State For every $r \in [c]$ sample $\boldsymbol{\alpha}^{r} \stackrel{\$}{\leftarrow}[N]^{\tau}$ and $\boldsymbol{\beta}^{r} \stackrel{\$}{\leftarrow} [\mathbb{Z}_{p}]^{\tau}$ uniformly at random.
\State For $r \in [c]$, sample FSS keys:
\begin{align*}
& \left(K_{0}^{r}, K_{1}^{r}\right) \stackrel{\$}{\leftarrow} \texttt{DSPF}^{\tau}_{N}\texttt{.Gen}\left(\mathds{1}^{\lambda}, \boldsymbol{\alpha}^{r}, v \cdot \boldsymbol{\beta}^{r}\right).
\end{align*}
\State Set $k_0 = (v, \{K_0^{r}\}_{r\in[c]})$ and $k_1 = (\{\boldsymbol{\alpha}^r, \boldsymbol{\beta}^r\}_{r\in[c]}, \{K_1^{r}\}_{r\in[c]})$.
\State Output $(k_0, k_1)$.
\end{algorithmic}

\vspace{1em} % Space before the next part

PCG.Expand$_{\texttt{VOLE}}(\sigma, k_\sigma)$:

\begin{algorithmic}[1]
\State If $\sigma=0$, parse $k_0$ as $(v, \{K_0^{r}\}_{r\in[c]})$
\State If $\sigma=1$, parse $k_1 $ as $(\{\boldsymbol{\alpha}^r, \boldsymbol{\beta}^r\}_{r\in[c]}, \{K_\sigma^{r}\}_{r\in[c]})$ and define for $r \in [c]$ the degree $< N$, $\tau$-sparse polynomials
\begin{align*}
e^r(X) = \sum_{k\in [t]} \boldsymbol{\beta}^r[k] \cdot X^{\boldsymbol{\alpha}^r[k]}
\end{align*}
to compose all $e^r$ to a single length-$c$ vector $\boldsymbol{e}$.
\State For each $i \in [c]$ compute vector $\boldsymbol{u}_\sigma$ with each element being a degree $< 2N$ polynomial:
\begin{align*}
& \boldsymbol{u}_\sigma[i] \leftarrow \texttt{DSPF}^{\tau}_{N}\texttt{.FullEval}(\sigma, K_\sigma^{i}) 
\texttt{ mod } F(X).
\end{align*}
\State Compute $z_\sigma  = \langle\boldsymbol{a},\boldsymbol{u}_\sigma\rangle$ \texttt{mod} $F(X)$.
\State If $\sigma=0$ compute $x_0 = \langle\boldsymbol{a},\boldsymbol{e}\rangle$ \texttt{mod} $F(X)$ and output $(x_0, z_0)$.
\State If $\sigma=1$ output $(v, z_1)$.
\end{algorithmic}
\end{specialconstruction}

\vspace{1em}

\textbf{Seed Generation.} For PCG.Gen$_{\texttt{VOLE}}$, we now only sample $c$ $\tau$-sparse LPN error polynomials for one party. The other party is assigned the constant term $v$. Instead of calculating the polynomial components' outer sum and tensor product, multiplying the coefficients (or non-zero elements) is sufficient to define the resulting point function. The seeds are generated accordingly, containing their respective shares of the point function $\{K_\sigma^{r}\}_{r\in[c]}$. For party $\sigma=0$, the LPN error polynomial is included, while for party $\sigma=1$, the constant is included, such that $k_0 = (v, \{K_0^{r}\}_{r\in[c]})$ and $k_1 = (\{\boldsymbol{\alpha}^r, \boldsymbol{\beta}^r\}_{r\in[c]}, \{K_1^{r}\}_{r\in[c]})$. 
\\\\
\textbf{Seed Expansion.} In PCG.Expand$_{\texttt{VOLE}}$, the parties evaluate their respective share $\boldsymbol{u}_\sigma$ of the multiplication by performing a full-domain evaluation of the multi-point function and computing the inner product of $\boldsymbol{u}_\sigma$ and the public input $\boldsymbol{a}$ results in $z_\sigma$. Party $\sigma=0$ must do the same for its LPN error vectors to learn $x_0$. The functionality returns $(x_0, z_0)$ to party $\sigma=0$ and $(v, z_0)$ to party $\sigma=1$. Similar to the construction for PCG for OLE, this yields a single VOLE over $R_p$. 
\\\\
In terms of security, the following derives from Theorem \ref{theorem:olepcg}: 
\begin{theorem}
Suppose the $R^c$-LPN$_\tau$ assumption holds and \texttt{DSPF} is a secure FSS scheme as formalized in Definition \ref{def:FSS}. Then, Construction 2 is a secure PCG for VOLE in $R_p$.
\end{theorem}

\subsubsection{Splitting the Ring}
Given a single shared VOLE correlation over $R_p$ through $(x_0, z_0)$ and $(v, z_0)$, the parties can again leverage the isomorphism between $R_p$ and $\mathbb{Z}_{p}$ to split the ring into $n$ individual VOLE correlations in $\mathbb{Z}_{p}$, similar to the approach outlined in Section \ref{subseq:realtiontofx}. The ring is split by evaluating the tuples on the roots of unity of $F(X)$. For any root of unity $\xi_i \in \{\xi_0, ...,\xi_{N-1}\}$ of $F(X)$
\begin{equation}
x_0(\xi_i) \cdot v = z_0(\xi_i)+ z_1(\xi_i)
\end{equation}
is the $i$-th VOLE correlation in $\mathbb{Z}_{p}$.

\subsubsection{Reduced Computational Complexity}
Compared to the complexity of the PCG for OLE, the VOLE construction offers a significant reduction in complexity: PCG.Gen$_{\texttt{VOLE}}$ is now in $O(ct)$, which is a clear improvement over the previous quadratic complexity. Further, the expansion PCG.Expand$_{\texttt{VOLE}}$ is reduced to $O(N)$, enabled by the efficient use of naive multiplication (over FFT) for $z_\sigma$. Although counter-intuitive, this is possible because the dense, degree-$N$ polynomials in $\boldsymbol{a}$ are multiplied by $\tau$-sparse polynomials in $\boldsymbol{e}$, requiring only $\tau \cdot N$ integer multiplications. Notice that the DSPF now operates with only $\tau$ special positions and a smaller domain $N$. While this doesn't change computational complexity, it has a noticeable impact on the construction runtime. We explore this in more detail in Section \todo X.