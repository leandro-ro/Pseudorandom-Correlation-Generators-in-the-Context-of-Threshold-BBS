\section*{Abstract}
Secure multi-party computation (MPC) protocols often employ correlated randomness to improve efficiency and benefit from this by separating their operations into a computationally intensive preprocessing (offline) phase and a highly efficient, non-interactive (online) phase. This trade-off is particularly valuable in environments dealing with high latency, as the online phase requires no further communication between the participating parties. In addition, it allows parties to utilize previously unused idle time for preprocessing in order to speed up the time-critical, input-dependent part of their protocol. Pseudorandom correlation generators (PCG) introduced by Boyle et al. \cite{boyle2019efficient, boyle2020efficient} facilitate the preprocessing approach by extending short correlated seeds into long instances of a target correlation. In particular, the primitive allows seed generation with sublinear communication complexity, while the extension can be performed locally, making the preprocessing phase practical compared to previous approaches. Despite their theoretical value and use in various MPC protocols, PCG implementations have been lacking. 

This work addresses this, by presenting practical considerations for implementing the PCG construction by Boyle et al. \cite{boyle2020efficient}. We demonstrate the value of these optimizations within the offline phase of Faust et al.'s non-interactive threshold signature BBS+ protocol \cite{faust2023non}. Our implementation is the first to support the threshold setting and achieves quasilinear runtime depending on the number of presignatures generated, as well as linear runtime increase with the number of participants. In the $2$-out-of-$2$ setting, our PCG generates $2^{19}$ presignatures in $179$ms per presignature, significantly outperforming the (only) other available implementation. 