\chapter{Introduction}
Correlated secret randomness is considered to be a useful resource for enhancing both the performance and security of secure multi-party computation (MPC) protocols \cite{ishai2013power}. The type of correlation determines the kind of MPC circuit supported; for example, oblivious transfer (OT) correlations support boolean circuits \cite{goldreich2019play}, while oblivious linear-function evaluation (OLE) is used for arithmetic circuits \cite{ishai2009secure}. In order to generate correlated secret randomness there is usually a distribution phase. In this phase, each party receives a sample from a joint random distribution. Although the samples are random, they adhere to the specified correlation. MPC protocols utilizing correlated secret randomness are therefore split up into an input-independent \textit{offline} and input-dependent \textit{online phase}. This is commonly referred to as \textit{preprocessing}, since the offline phase generates many instances of the required correlation, which are then consumed by the online phase. The approach has the potential to make the online phase very efficient by shifting communication and computationally demanding tasks to the offline phase. Preprocessing is therefore very appealing for real-world applications, since it enables participants to utilize their idle times to speed up the critical (on-demand) part of their MPC protocol. This flexibility can also lead to significant cost savings when using elastic cloud computing environments \cite{coutinho2015elasticity}, where providers offer substantial discounts to users willing to use spare computing capacity during off-peak periods\footnote{AWS Spot Instances: \url{https://aws.amazon.com/aws-cost-management/aws-cost-optimization/spot-instances/} \\ Azure Spot Instances: \url{https://azure.microsoft.com/en-us/products/virtual-machines/spot/}}.
\\\\
\textbf{Pseudorandom Correlation Generators.} In recent years, Boyle et al. proposed and further refined the novel primitive of a pseudorandom correlation generator (PCG) \cite{boyle2015function, boyle2016function, boyle2020efficient}, which provides a promising approach for the realization of a preprocessing phase. A PCG can be thought of as a form of (distributed) pseudorandom generator in which participants generate seeds and then locally expand them for pseudorandom bit streams. Crucially, the PCG expansion ensures that the generated bit streams are correlated across participants. Such a primitive provides several advantages to the preprocessing model: It reduces offline communication because the seed expansion is done locally. It also reduces storage costs, since the correlations are compressed in the PCG seed and the parties can decide to expand them only when needed. Furthermore, only the seed generation needs to be protected against malicious parties, since malicious seed expansion does not effect honest parties. The most practical PCG construction proposed by Boyle et al. \cite{boyle2020efficient}, is based on the Learning Parity with Noise (LPN) assumption \cite{pietrzak2012cryptography}. The construction is of high practical relevance as it avoids the quadratic complexity of previous approaches. The PCG realizes OLE and Vector-OLE correlations, but the general approach can be modified to produce multiplication and authentication triples.
\\\\
\textbf{Application for Threshold Signature Schemes.}  Threshold signature schemes \cite{desmedt1987society, desmedt1992threshold} distribute the ability to generate a digital signature among multiple participants (signers). In practice, any subset of $t$ signers can collaborate to produce a signature, while the scheme remains secure even if up to $t - 1$ signers are compromised. This eliminates single points of failure and increases security. The rise of blockchain technology has led to increasing interest in threshold signature schemes for use cases such as securing digital wallets \cite{gennaro2016threshold} and building distributed, anonymous credentialing systems \cite{garman2013decentralized}. However, many threshold signature schemes require interaction between signers during the signing process. This interaction results in high latency, especially when the parties are geographically dispersed. For example, consider two parties located in Western Europe and the West Coast of the United States; Azure network latency alone can reach 147ms \footnote{\url{https://learn.microsoft.com/en-us/azure/networking/azure-network-latency} (accessed April 2, 2024)}. Geographic distance therefore creates a significant performance bottleneck for interactive threshold signature schemes, which worsens with each additional round of communication required. To mitigate this shortcoming through preprocessing, PCGs have recently received attention for use in threshold signature schemes \cite{abram2022low, cryptoeprint:2023/1076}. These schemes use the PCG primitive to generate message-independent pre-signatures in the offline phase, so that the actual signing can be facilitated in the online phase without interaction between the signing parties. The threshold BBS+ scheme proposed by Faust et al. \cite{cryptoeprint:2023/1076} is of particular interest because it supports non-interactive signing in the online phase while keeping the communication complexity in the offline phase sublinear. Both attributes have not been achieved simultaneously by other schemes such as threshold ECDSA \cite{abram2022low} or multi-party Schnorr signatures \cite{kondi2023two}. 

\section{Our Contribution}
This thesis focuses on the efficient implementation and evaluation of the LPN-based PCG proposed by Boyle et al. \cite{boyle2020efficient}. We present a theoretical and practical analysis of their PCG construction and evaluate its practicality by realize the offline phase of Faust et al.'s threshold BBS+ scheme. We summarize our contribution as follows:

\begin{itemize}
\item \textbf{PCG Derivation} In Chapter \ref{chapter:PCGforVOLE} we provide a step-by-step derivation of the PCG by Boyle et al. that leads to both OLE and Vector-OLE constructions. In this context, we present and prove a formula for efficiently generating so-called \textit{roots of unity} by iteration.
\item \textbf{Practical Considerations:} In Chapter \ref{chapter:ImplementingPCGs}, we identify potential bottlenecks in the PCG's underlying building blocks and propose practical optimizations to improve their performance in this specific setting.
\item \textbf{Benchmarking and Evaluation:} In Chapter \ref{chapter:evaluation}, we provide thorough benchmarks of the optimized building blocks and then proceed by evaluating the overall performance of our PCG implementation for both OLE and Vector-OLE.
\item \textbf{Implementing a PCG for BBS+:} In Chapter \ref{chapter:PCGforBBSPlus}, we demonstrate the practicality of Boyle et al.'s PCG by proposing a construction that realizes the offline phase of Faust et al.'s threshold BBS+ scheme \cite{cryptoeprint:2023/1076}. We implement this construction and evaluate its performance, providing insights into both our implementation efficiency and the implications for the scheme's online phase.
\end{itemize}

\section{Related Work}
Most related to our work is the implementation provided for the threshold ECDSA scheme, that also utilizes the PCG provided by Boyle et al. As the focus of their work is on the proposed threshold ECDA scheme, they provide only rudementary details regarding the implememtation, ase well as only limited benchmarks, which we compare our construction to in Section \ref{subsec:bbspPcgBottlenecks}. Other than this, there are no PCG implementations known to us.