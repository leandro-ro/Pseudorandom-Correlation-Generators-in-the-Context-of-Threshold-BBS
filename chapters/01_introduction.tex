\chapter{Introduction}
Correlated secret randomness is considered to be a useful resource for enhancing both the performance and security of secure multi-party computation (MPC) protocols \cite{ishai2013power}. The type of correlation determines the kind of MPC circuit supported; for example, oblivious transfer (OT) correlations support boolean circuits \cite{goldreich2019play}, while oblivious linear-function evaluation (OLE) is used for arithmetic circuits \cite{ishai2009secure}. In order to get correlated secret randomness there is usually a distribution phase. In this phase, each party receives a sample from a joint random distribution. Although the samples are random, they adhere to the specified correlation. MPC protocols utilizing correlated secret randomness are therefore split up into an input-independent \textit{offline} and input-dependent \textit{online phase}. This is commonly referred to as \textit{preprocessing}, since the offline phase generates many instances of the required correlation, which are then consumed by the online phase. The approach has the potential to make the online phase very efficient by shifting communication and computationally demanding tasks to the offline phase. Preprocessing is therefore very appealing for real-world applications, since it enables participants to utilize their idle times to speed up the critical (on-demand) part of their MPC protocol. This flexibility can also lead to significant cost savings in elastic cloud computing environments \cite{coutinho2015elasticity}, where providers often offer substantial discounts for users willing to utilizing spare computational capacity\footnote{AWS Spot Instances: \url{https://aws.amazon.com/aws-cost-management/aws-cost-optimization/spot-instances/}}\footnote{Azure Spot Instances: \url{https://azure.microsoft.com/en-us/products/virtual-machines/spot/}}.
\\\\
\textbf{Pseudorandom Correlation Generators.} In recent years, Boyle et al. proposed and further refined the novel primitive of a pseudorandom correlation generator (PCG) \cite{boyle2015function, boyle2016function, boyle2020efficient}, which provides a promising approach for the realization of a preprocessing phase. A PCG can be thought of as a form of (distributed) pseudorandom generator in which participants generate seeds and then locally expand them for pseudorandom bit streams. Crucially, the PCG expansion ensures that the generated bit streams are correlated across participants. Such a primitive provides several advantages to the preprocessing model: On the one hand, it reduces offline communication because the seed expansion is done locally. It also reduces storage costs, since the correlations are compressed in the PCG seed and the parties can decide to expand them only when needed. Furthermore, only the seed generation needs to be protected against malicious parties, since malicious seed expansion does not effect honest parties.
\\\\
This work implements and evaluates an efficient PCG construction proposed by Boyle et al. \cite{boyle2020efficient}, which is based on the Learning Parity with Noise (LPN) assumption. The construction is of high practical relevance because it avoids the quadratic complexity of previous approaches. The PCG realizes OLE and Vector-OLE correlations, but the general approach can be modified to produce multiplication and authentication triples. Due to the efficiency and versatility of this PCG, it has recently received attention for use in threshold signature schemes \cite{abram2022low, cryptoeprint:2023/1076}. These schemes utilize the PCG primitive to generate message-independent pre-signatures in the offline phase, so that the actual signing in the online phase can be facilitated without interaction between the signing parties. The Threshold BBS+ scheme of Faust et al. \cite{cryptoeprint:2023/1076} is of particular interest because it supports non-interactive signing in the online phase while keeping the communication complexity in the offline phase sublinear. This dual achievement has not been simultaneously achieved by other schemes such as threshold ECDSA \cite{abram2022low} or Schnorr \cite{kondi2023two}. We apply the lessons learned from our implementation of Boyle et al.'s PCG primitive to implement a combined PCG that realizes the offline phase for Faust et al.'s Threshold BBS+ scheme. We evaluate the schemes overall performance in the offline phase and derive implications for the online phase.

\section{Contribution}


\section{Related Work}
\section{Outline}