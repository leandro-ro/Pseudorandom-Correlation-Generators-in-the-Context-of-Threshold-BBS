\chapter{Introduction}
Correlated secret randomness is considered to be a useful resource for enhancing both the performance and security of secure multi-party computation (MPC) protocols \cite{ishai2013power}. The type of correlation determines the kind of MPC circuit supported; for example, oblivious transfer (OT) correlations work with boolean circuits \cite{goldreich2019play}, while oblivious linear-function evaluation (OLE) is used for arithmetic circuits \cite{ishai2009secure}. In order to get correlated secret randomness there is usually a distribution phase. In this phase, each party receives a sample from a joint random distribution. Although the samples are random, they adhere to the specified correlation. MPC protocols utilizing correlated secret randomness are therefore split up into an input-independent \textit{offline} and input-dependent \textit{online phase}. This is commonly referred to as \textit{preprocessing}, since the offline phase generates many instances of the required correlation, which are then consumed by the online phase. The approach has the potential to make the online phase very efficient by shifting communication and computationally demanding tasks to the offline phase. Preprocessing is therefore very appealing for real-world applications, since it enables participants to utilize their idle times to speed up the critical (on-demand) part of their MPC protocol.
\\\\
In recent years, Boyle et al. proposed \cite{boyle2015function} and further refined \cite{boyle2016function, boyle2020efficient} the novel primitive of a pseudorandom correlation generator (PCG), which provides a promising approach for the realization of a preprocessing phase. A PCG can be thought of as a form of (distributed) pseudorandom generator in which participants generate seeds and then locally expand them for pseudorandom bit streams. Crucially, the PCG expansion ensures that the generated bit streams are correlated across participants. Such a primitive provides several advantages to the preprocessing model: On the one hand, it reduces offline communication because seed expansion is done locally. It also reduces storage costs, since the correlations are compressed in the PCG seed and the parties can decide to expand them only when needed. Furthermore, only the seed generation needs to be protected against malicious parties, since malicious seed expansion does not effect honest parties.
\\\\
This work implements and evaluates an efficient \texttt{Ring-LPN}-based PCG construction proposed by Boyle et al. in \cite{boyle2020efficient}. The construction is of high practical relevance since it avoids the quadratic complexity of previous approaches. The PCG realizes OLE and Vector-OLE correlations, but the general approach can be modified to produce multiplication and authentication triples.


\\\\
% BBS+
Multi-Party Computation (MPC) is commonly used in modern cryptographic schemes, enabling multiple parties to jointly compute a function over their inputs while keeping those inputs private. An important concept within MPC is correlated secret randomness, a mechanism that, despite its simplicity, significantly enhances the efficiency and feasibility of lightweight MPC protocols, especially in scenarios lacking an honest majority. Correlated randomness, such as oblivious transfer and multiplication triples, underpins secure computation by pre-establishing secret, shared structures that simplify and streamline the online computation phase.
\\\\
The preprocessing model emerges as a natural evolution in the pursuit of practical MPC, addressing the inherent challenges of real-time computation under stringent security requirements. Within this model, the concept of Pseudorandom Correlation Generators (PCGs) introduces a groundbreaking approach. PCGs enable parties to generate a compact seed in a setup phase that can later be expanded into a large volume of correlated randomness needed for secure computation, without further interaction. This paradigm not only promises to reduce the communication and computational overhead associated with traditional MPC protocols but also opens new avenues for efficiency and scalability in cryptographic designs.
\\\\
As an applied manifestation of these theoretical advancements, this work focuses on realizing PCGs for a specific non-interactive Threshold BBS+ scheme. The BBS+ signature scheme, lauded for its application in anonymous credential systems, stands out for its constant-size signatures over a set of attributes and its efficient protocols for both blind signing and selective disclosure. These features make BBS+ an invaluable tool in preserving privacy in digital interactions, where the ability to verify certain attributes without revealing a user's identity is increasingly crucial. By thresholding BBS+ signatures, we distribute the signing capability across multiple parties, eliminating single points of failure and enhancing security. Moreover, adopting a non-interactive approach during the signing phase significantly elevates the scheme's practicality, reducing latency and communication overhead in distributed environments.
\\\\
This work leverages the PCG framework proposed by Boyle et al., implementing it to produce the requisite correlated randomness for OLE and VOLE correlations. Subsequently, these implementations serve as the foundation for constructing a PCG for the BBS+ scheme, embodying a stride forward in the practical application of threshold cryptography and MPC in securing digital credentials. While Boyle et al. provided the theoretical blueprint, our contribution lies in bringing this vision to fruition, offering a tangible, efficient mechanism for generating correlated randomness that underpins a non-interactive, thresholded BBS+ signature scheme.

\section{Contribution}
\section{Related Work}
\section{Outline}