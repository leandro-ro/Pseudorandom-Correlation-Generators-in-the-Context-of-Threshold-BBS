\chapter{Conclusion}
The central contribution of this thesis is the presentation of practical considerations for implementing Boyle et al.'s \texttt{Ring-LPN} based PCG primitive \cite{boyle2020efficient}. These considerations include the optimization of the underlying DSPF building block for reduced space complexity (Section \ref{sec:dspfImplementation}) and the strategic use of different approaches for handling sparse polynomials (Section \ref{sec:polyOperationsImpl}). We also presented a formula for iteratively computing roots of unity (Section \ref{subseq:realtiontofx}) and show how the principles of Horner's method \cite{horner1819xxi} can be used to compute all roots of unity within linear time complexity (Section \ref{sec:fxconsiderationsImpl}). Evaluations within Chapter \ref{chapter:evaluation} highlight the impact of these optimizations on the practicality of the PCG construction. The PCG implementation derived from the optimized building blocks exhibits quasilinear scaling as the number of correlations generated increases (Section \ref{subsec:evalExpansionVOLE}). This is very appealing for real-world applications that utilize this PCG for preprocessing since the setup phase amortizes faster, and it allows the non-interactive online phase to be longer by mitigating the repetition of the preprocessing phase.
\\\\
To further demonstrate the value of our optimizations, we incorporate the building blocks for implementing the preprocessing phase of Faust et al.'s threshold BBS+ scheme \cite{faust2023non}. Our proposed PCG is optimized for the $n$-out-of-$n$ case (Section \ref{subsec:pcgForBBs+}) and can easily adapt to the threshold setting (Section \ref{subsec:tauoutofnSetting}). Implementations for both cases validate the quasilinear scaling concerning the number of preprocessed correlations (Section \ref{sec:evalBBSPlusPCG}). Notably, the preprocessing phase exhibits linear scaling concerning the number of parties, demonstrating the scheme's suitability for including many signers. Compared to the only other PCG implementation available \cite{abram2022low} (which is limited to the $n$-out-of-$n$ case), our implementation achieves a 6x performance improvement for $n=2$ (Section \ref{subsec:bbspPcgBottlenecks}), clearly highlighting the effectiveness of our practical considerations considering that our PCG includes an additional OLE correlation.
\\\\
Finally, within the $t$-out-of-$n$ threshold case, we acknowledge a limitation of the proposed PCG for threshold BBS+ signature scheme, which introduces a linear overhead in the online phase proportional to the number of precomputed correlations (Section \ref{subsec:implNIBBs+}). We observe that this overhead significantly impacts the protocol runtime in low-latency environments (LAN), while its impact is insignificant in high-latency environments (WAN).

\section{Future Work}
We identify several promising directions for future research. Firstly, Boyle et al. \cite{boyle2020efficient} propose various suitable LPN parameter sets achieving 128-bit security equivalence. While we employed $(c,\tau)=(4,16)$ for consistency with prior work \cite{abram2022low}, evaluating performance across different parameter choices for potentially higher security levels presents an interesting direction for future practical assessment. Secondly, our work primarily addressed PCG expansion, assuming a trusted seed generation phase. Implementing and evaluating the distributed setup phase outlined (but not implemented) by Abram et al. \cite{abram2022low} would be a valuable extension. Furthermore, in Section \ref{subsec:toutofnEval}, we briefly mention the possibility to reduce the overhead introduced within the $t$-out-of-$n$ threshold setting. Optimizations here could mitigate the overhead that the offline phase introduces to the online phase, making the scheme more interesting for low-latency settings as well. Finally, Tessaro et al. \cite{tessaro2023revisiting} recently proposed a more compact BBS+ signature scheme that eliminates the need for one OLE correlation within Faust et al.'s BBS+ presignatures. Investigating the potential performance gains of applying this simplification to the PCG remains future work.